% Don't like 10pt? Try 11pt or 12pt
\documentclass[10pt]{article}

\usepackage{calc}
% Simpler bibsection for CV sections
% (thanks to natbib for inspiration)
\makeatletter
\newlength{\bibhang}
\setlength{\bibhang}{1em}
\newlength{\bibsep}
 {\@listi \global\bibsep\itemsep \global\advance\bibsep by\parsep}
\newenvironment{bibsection}%
        {\vspace{-\baselineskip}\begin{list}{}{%
       \setlength{\leftmargin}{\bibhang}%
       \setlength{\itemindent}{-\leftmargin}%
       \setlength{\itemsep}{\bibsep}%
       \setlength{\parsep}{\z@}%
        \setlength{\partopsep}{0pt}%
        \setlength{\topsep}{0pt}}}
        {\end{list}\vspace{-.6\baselineskip}}
\makeatother

% Layout: Puts the section titles on left side of page
\reversemarginpar


%% Use these lines for A4-sized paper
\usepackage[paper=a4paper,
            %includefoot, % Uncomment to put page number above margin
            marginparwidth=30.5mm,    % Length of section titles
            marginparsep=1.5mm,       % Space between titles and text
            margin=25mm,              % 25mm margins
            includemp]{geometry}

%% More layout: Get rid of indenting throughout entire document
\setlength{\parindent}{0in}

%% This gives us fun enumeration environments. compactitem will be nice.
\usepackage{paralist}

%% Reference the last page in the page number
%
% NOTE: comment the +LP line and uncomment the -LP line to have page
%       numbers without the ``of ##'' last page reference)
%
% NOTE: uncomment the \pagestyle{empty} line to get rid of all page
%       numbers (make sure includefoot is commented out above)
%
\usepackage{fancyhdr,lastpage}
\pagestyle{fancy}
%\pagestyle{empty}      % Uncomment this to get rid of page numbers
\fancyhf{}\renewcommand{\headrulewidth}{0pt}
\fancyfootoffset{\marginparsep+\marginparwidth}
\newlength{\footpageshift}
\setlength{\footpageshift}
          {0.5\textwidth+0.5\marginparsep+0.5\marginparwidth-2in}
\lfoot{\hspace{\footpageshift}%
       \parbox{4in}{\, \hfill %
                    \arabic{page} of \protect\pageref*{LastPage} % +LP
%                    \arabic{page}                               % -LP
                    \hfill \,}}

% Finally, give us PDF bookmarks
\usepackage{color,hyperref}
\definecolor{darkblue}{rgb}{0.0,0.0,0.3}
\hypersetup{colorlinks,breaklinks,
            linkcolor=darkblue,urlcolor=darkblue,
            anchorcolor=darkblue,citecolor=darkblue}

%%%%%%%%%%%%%%%%%%%%%%%% End Document Setup %%%%%%%%%%%%%%%%%%%%%%%%%%%%


%%%%%%%%%%%%%%%%%%%%%%%%%%% Helper Commands %%%%%%%%%%%%%%%%%%%%%%%%%%%%

% The title (name) with a horizontal rule under it
%
% Usage: \makeheading{name}
%
% Place at top of document. It should be the first thing.
\newcommand{\makeheading}[1]%
        {\hspace*{-\marginparsep minus \marginparwidth}%
         \begin{minipage}[t]{\textwidth+\marginparwidth+\marginparsep}%
                {\large \bfseries #1}\\[-0.15\baselineskip]%
                 \rule{\columnwidth}{1pt}%
         \end{minipage}}

% The section headings
%
% Usage: \section{section name}
%
% Follow this section IMMEDIATELY with the first line of the section
% text. Do not put whitespace in between. That is, do this:
%
%       \section{My Information}
%       Here is my information.
%
% and NOT this:
%
%       \section{My Information}
%
%       Here is my information.
%
% Otherwise the top of the section header will not line up with the top
% of the section. Of course, using a single comment character (%) on
% empty lines allows for the function of the first example with the
% readability of the second example.
\renewcommand{\section}[2]%
        {\pagebreak[3]\vspace{1.3\baselineskip}%
         \phantomsection\addcontentsline{toc}{section}{#1}%
         \hspace{0in}%
         \marginpar{
         \raggedright \scshape #1}#2}

% An itemize-style list with lots of space between items
\newenvironment{outerlist}[1][\enskip\textbullet]%
        {\begin{itemize}[#1]}{\end{itemize}%
         \vspace{-.1\baselineskip}}

% An environment IDENTICAL to outerlist that has better pre-list spacing
% when used as the first thing in a \section
\newenvironment{lonelist}[1][\enskip\textbullet]%
        {\vspace{-.1\baselineskip}\begin{list}{#1}{%
        \setlength{\partopsep}{0pt}%
        \setlength{\topsep}{0pt}}}
        {\end{list}\vspace{-.1\baselineskip}}

% An itemize-style list with little space between items
\newenvironment{innerlist}[1][\enskip\textbullet]%
        {\begin{compactitem}[#1]}{\end{compactitem}}

% An environment IDENTICAL to innerlist that has better pre-list spacing
% when used as the first thing in a \section
\newenvironment{loneinnerlist}[1][\enskip\textbullet]%
        {\begin{compactitem}[#1]}
        {\end{compactitem}\vspace{-.1\baselineskip}}

% To add some paragraph space between lines.
% This also tells LaTeX to preferably break a page on one of these gaps
% if there is a needed pagebreak nearby.
\newcommand{\blankline}{\quad\pagebreak[2]}

% Uses hyperref to link DOI
\newcommand\doilink[1]{\href{http://dx.doi.org/#1}{#1}}
\newcommand\doi[1]{doi:\doilink{#1}}

% For \url{SOME_URL}, links SOME_URL to the url SOME_URL
\providecommand*\url[1]{\href{#1}{#1}}
% Same as above, but pretty-prints SOME_URL in teletype fixed-width font
\renewcommand*\url[1]{\href{#1}{\texttt{#1}}}

% For \email{ADDRESS}, links ADDRESS to the url mailto:ADDRESS
\providecommand*\email[1]{\href{mailto:#1}{#1}}
% Same as above, but pretty-prints ADDRESS in teletype fixed-width font
%\renewcommand*\email[1]{\href{mailto:#1}{\texttt{#1}}}

%%%%%%%%%%%%%%%%%%%%%%%% End Helper Commands %%%%%%%%%%%%%%%%%%%%%%%%%%%




\usepackage{fancyhdr}
\usepackage{datetime}

\pagestyle{fancy}
\fancyfoot{}
\fancyfoot[LO]{\small \today\ \currenttime}

%%%%%%%%%%%%%%%%%%%%%%%%% Begin CV Document %%%%%%%%%%%%%%%%%%%%%%%%%%%%

\begin{document}
\makeheading{Laurent A.F. Callot}
%
% NOTE: Mind where the & separators and \\ breaks are in the following
%       table.
%
% ALSO: \rcollength is the width of the right column of the table
%       (adjust it to your liking; default is 1.85in).
%
\section{Contact Information}

\newlength{\rcollength}\setlength{\rcollength}{5cm}%
\newlength{\rcollengthb}\setlength{\rcollengthb}{5cm}%
%
\begin{tabular}[t]{@{}p{\textwidth-\rcollength}p{\rcollength}}
\textit{E-mail:} \email{l.callot@vu.nl}& \textit{Phone:} {+31 642 50 82 94}\\
\textit{Website: }\href{www.lcallot.github.io}{lcallot.github.io}\\
\end{tabular}

\section{Current Positions}

\begin{tabular}[t]{@{}p{\textwidth-\rcollengthb}p{\rcollengthb}}
\textbf{Post-Doctoral Researcher}	& \href{http://www.vu.nl/}{VU University Amsterdam.}\\
\href{http://www.feweb.vu.nl}{Department of Econometrics and OR.} &\\
November 2012 to present.	&
\end{tabular}
%

\section{Other Affiliations}
\href{http://www.creates.au.dk}{CREATES}, and \href{http://www.tinbergen.nl}{the Tinbergen Institute}.

\section{Research Interests}
%
Time series econometrics, High dimensional statistics, international macroeconomics and finance.

\section{Education}
%
\textbf{Aarhus University},
\begin{outerlist}


\item[] PhD Economics, September 2012
        \begin{innerlist}
	\item[] Thesis title: \emph{Large Panels and High Dimensional VARs.}
        \item[] Adviser: Professor Niels Haldrup.
	\item[] Committee: Eric Hillebrant (Aarhus University), Anders Rahbek (Copenhagen University), Patrick Groenen (Erasmus University). 
        \end{innerlist}

\end{outerlist}

\textbf{Princeton University},
\begin{outerlist}


\item[]	Visiting Student Research Collaborator, September-December 2011. Sponsors: Professor Bo Honor\'{e} and Ulrich M\"{u}ller.

\end{outerlist}

\textbf{Aarhus University},
\begin{outerlist}


\item[] M.Sc. Economics (\textit{cand.oecon}), August 2009.
        \begin{innerlist}
	\item[] Thesis title: \emph{Modelling Exchange rates with Global VARs.}
	\item[] Adviser: Professor Niels Haldrup.

        \end{innerlist}

\end{outerlist}

\textbf{University Paris X},
\begin{outerlist}

\item[] B.Sc. Economics (Licence Sciences Economiques), June 2007.


\end{outerlist}

\textbf{University Paris VI},
\begin{outerlist}

\item[] B.Sc. Mathematics (Licence de Sciences et Technologies, Mention mathematiques), January 2007.
\item[] DEUG MIAS (Mathematics Informatiques et Applications aux Sciences), June 2004.
\end{outerlist}




\vspace{0.1in}


\section{Publications}

The latest version of the every paper with replication files is available on my website.

\begin{outerlist}
\item[] \textbf{Refereed Publications}\\
    \begin{innerlist}
\item[]Callot, Laurent A.F and Kock, Anders Bredahl: {Oracle Efficient Estimation and Forecasting with the Adaptive Lasso and the Adaptive Group Lasso in Vector Autoregressions}, forthcoming in \textit{Essays in Nonlinear Time Series Econometrics.} Oxford University Press.
\item[]Callot, Laurent A.F., Paldam, Martin: {Natural funnel asymmetries. A simulation analysis of the three basic tools of meta analysis}, \textit{Research Synthesis Methods Volume 2, Issue 2, pages 84–102, June 2011}
    \end{innerlist}

\item[] \textbf{Working papers}\\
    \begin{outerlist}
\item[]Kock, Anders Bredahl and Callot, Laurent A.F: {Oracle Inequalities for High Dimensional Vector Autoregressions}. \textit{Submitted after revision to the Journal of Econometrics}.
\item[]Callot, Laurent A.F: {A Bootstrap Co-integration Rank Test for Panels of VAR Models}, \textit{CREATES working paper}
\item[]Callot, Laurent A.F: {Estimating and Testing for a Common Co-integration Space in Large Panel VAR}, \textit{CREATES working paper}
    \end{outerlist}
\end{outerlist}



\section{Teaching Experience}
\begin{outerlist}
\item[] \textit{Lecturer}\\
    \begin{innerlist}
    	\item[]Introductory Econometrics, Spring 2013, VU Amsterdam.
        \item[]Introduction to programming, August/September 2011, Aarhus University.
        \blankline
    \end{innerlist}

\item[] \textit{Teaching Assistant}\\
    \begin{innerlist}
	\item[]Econometrics, Fall 2010 , Aarhus University.
	\item[]Advanced Programming in Quantitative Economics, August 2010 , Aarhus University.
        \item[]Macro 1, Fall 2009 , Aarhus University.
        \blankline
    \end{innerlist}


\item[] \textit{Research Assistant}\\
    \begin{innerlist}
        \item[]Assistant to Prof. Martin Paldam, 2008 and 2009.
        \blankline
    \end{innerlist}

\end{outerlist}


\section{Refereeing}

    \begin{innerlist}
    \item[]Econometric Review.
    \item[]The Energy Journal ($\times 2$).
    \item[]Computational Statistics. 
        \blankline
    \end{innerlist}
 

\section{Languages}
%
\blankline
French (native), English (fluent), Danish (proficient), Spanish (basic).



\end{document}

%%%%%%%%%%%%%%%%%%%%%%%%%% End CV Document %%%%%%%%%%%%%%%%%%%%%%%%%%%%%
