% Don't like 10pt? Try 11pt or 12pt
\documentclass[10pt]{article}

\usepackage{calc}
% Simpler bibsection for CV sections
% (thanks to natbib for inspiration)
\makeatletter
\newlength{\bibhang}
\setlength{\bibhang}{1em}
\newlength{\bibsep}
 {\@listi \global\bibsep\itemsep \global\advance\bibsep by\parsep}
\newenvironment{bibsection}%
        {\vspace{-\baselineskip}\begin{list}{}{%
       \setlength{\leftmargin}{\bibhang}%
       \setlength{\itemindent}{-\leftmargin}%
       \setlength{\itemsep}{\bibsep}%
       \setlength{\parsep}{\z@}%
        \setlength{\partopsep}{0pt}%
        \setlength{\topsep}{0pt}}}
        {\end{list}\vspace{-.6\baselineskip}}
\makeatother

% Layout: Puts the section titles on left side of page
\reversemarginpar


%% Use these lines for A4-sized paper
\usepackage[paper=a4paper,
            %includefoot, % Uncomment to put page number above margin
            marginparwidth=30.5mm,    % Length of section titles
            marginparsep=1.5mm,       % Space between titles and text
            margin=25mm,              % 25mm margins
            includemp]{geometry}

%% More layout: Get rid of indenting throughout entire document
\setlength{\parindent}{0in}

%% This gives us fun enumeration environments. compactitem will be nice.
\usepackage{paralist}

%% Reference the last page in the page number
%
% NOTE: comment the +LP line and uncomment the -LP line to have page
%       numbers without the ``of ##'' last page reference)
%
% NOTE: uncomment the \pagestyle{empty} line to get rid of all page
%       numbers (make sure includefoot is commented out above)
%
\usepackage{fancyhdr,lastpage}
\pagestyle{fancy}
%\pagestyle{empty}      % Uncomment this to get rid of page numbers
\fancyhf{}\renewcommand{\headrulewidth}{0pt}
\fancyfootoffset{\marginparsep+\marginparwidth}
\newlength{\footpageshift}
\setlength{\footpageshift}
          {0.5\textwidth+0.5\marginparsep+0.5\marginparwidth-2in}
\lfoot{\hspace{\footpageshift}%
       \parbox{4in}{\, \hfill %
                    \arabic{page} of \protect\pageref*{LastPage} % +LP
%                    \arabic{page}                               % -LP
                    \hfill \,}}

% Finally, give us PDF bookmarks
\usepackage{color,hyperref}
\definecolor{darkblue}{rgb}{0.0,0.0,0.2}
\hypersetup{colorlinks,breaklinks,
            linkcolor=darkblue,urlcolor=darkblue,
            anchorcolor=darkblue,citecolor=darkblue}

%%%%%%%%%%%%%%%%%%%%%%%% End Document Setup %%%%%%%%%%%%%%%%%%%%%%%%%%%%


%%%%%%%%%%%%%%%%%%%%%%%%%%% Helper Commands %%%%%%%%%%%%%%%%%%%%%%%%%%%%

% The title (name) with a horizontal rule under it
%
% Usage: \makeheading{name}
%
% Place at top of document. It should be the first thing.
\newcommand{\makeheading}[1]%
        {\hspace*{-\marginparsep minus \marginparwidth}%
         \begin{minipage}[t]{\textwidth+\marginparwidth+\marginparsep}%
                {\large \bfseries #1}\\[-0.15\baselineskip]%
                 \rule{\columnwidth}{1pt}%
         \end{minipage}}

% The section headings
%
% Usage: \section{section name}
%
% Follow this section IMMEDIATELY with the first line of the section
% text. Do not put whitespace in between. That is, do this:
%
%       \section{My Information}
%       Here is my information.
%
% and NOT this:
%
%       \section{My Information}
%
%       Here is my information.
%
% Otherwise the top of the section header will not line up with the top
% of the section. Of course, using a single comment character (%) on
% empty lines allows for the function of the first example with the
% readability of the second example.
\renewcommand{\section}[2]%
        {\pagebreak[3]\vspace{1.3\baselineskip}%
         \phantomsection\addcontentsline{toc}{section}{#1}%
         \hspace{0in}%
         \marginpar{
         \raggedright \scshape #1}#2}

% An itemize-style list with lots of space between items
\newenvironment{outerlist}[1][\enskip\textbullet]%
        {\begin{itemize}[#1]}{\end{itemize}%
         \vspace{-.1\baselineskip}}

% An environment IDENTICAL to outerlist that has better pre-list spacing
% when used as the first thing in a \section
\newenvironment{lonelist}[1][\enskip\textbullet]%
        {\vspace{-.1\baselineskip}\begin{list}{#1}{%
        \setlength{\partopsep}{0pt}%
        \setlength{\topsep}{0pt}}}
        {\end{list}\vspace{-.1\baselineskip}}

% An itemize-style list with little space between items
\newenvironment{innerlist}[1][\enskip\textbullet]%
        {\begin{compactitem}[#1]}{\end{compactitem}}

% An environment IDENTICAL to innerlist that has better pre-list spacing
% when used as the first thing in a \section
\newenvironment{loneinnerlist}[1][\enskip\textbullet]%
        {\begin{compactitem}[#1]}
        {\end{compactitem}\vspace{-.1\baselineskip}}

% To add some paragraph space between lines.
% This also tells LaTeX to preferably break a page on one of these gaps
% if there is a needed pagebreak nearby.
\newcommand{\blankline}{\quad\pagebreak[2]}

% Uses hyperref to link DOI
\newcommand\doilink[1]{\href{http://dx.doi.org/#1}{#1}}
\newcommand\doi[1]{doi:\doilink{#1}}

% For \url{SOME_URL}, links SOME_URL to the url SOME_URL
\providecommand*\url[1]{\href{#1}{#1}}
% Same as above, but pretty-prints SOME_URL in teletype fixed-width font
\renewcommand*\url[1]{\href{#1}{\texttt{#1}}}

% For \email{ADDRESS}, links ADDRESS to the url mailto:ADDRESS
\providecommand*\email[1]{\href{mailto:#1}{#1}}
% Same as above, but pretty-prints ADDRESS in teletype fixed-width font
%\renewcommand*\email[1]{\href{mailto:#1}{\texttt{#1}}}

%%%%%%%%%%%%%%%%%%%%%%%% End Helper Commands %%%%%%%%%%%%%%%%%%%%%%%%%%%




\usepackage{fancyhdr}
\usepackage{datetime}

\pagestyle{fancy}
\fancyfoot{}
\fancyfoot[LO]{\small \today\ \currenttime}

%%%%%%%%%%%%%%%%%%%%%%%%% Begin CV Document %%%%%%%%%%%%%%%%%%%%%%%%%%%%

\begin{document}
\makeheading{Laurent A.F. Callot}
%
% NOTE: Mind where the & separators and \\ breaks are in the following
%       table.
%
% ALSO: \rcollength is the width of the right column of the table
%       (adjust it to your liking; default is 1.85in).
%
\section{Contact Information}

\newlength{\rcollength}\setlength{\rcollength}{5cm}%
\newlength{\rcollengthb}\setlength{\rcollengthb}{5cm}%
%
\begin{tabular}[t]{@{}p{\textwidth-\rcollength}p{\rcollength}}
\textit{E-mail:} \email{l.callot@vu.nl}& \textit{Phone:} {+31 642 50 82 94}\\
\textit{Website: }\href{http://lcallot.github.io}{lcallot.github.io}\\
\end{tabular}

\section{Current Positions}
\textbf{Post-Doctoral Researcher} (since November 2012)\\
\href{http://www.vu.nl/}{VU University Amsterdam}, \href{http://www.feweb.vu.nl}{Department of Econometrics and OR.}

\section{Other Affiliations}
Research Fellow: \href{http://www.tinbergen.nl}{Tinbergen Institute}, the Netherlands.\\
Junior Fellow: \href{http://www.creates.au.dk}{CREATES}, Aarhus University, Denmark.

\section{Research Interests}
%
Time series econometrics, High dimensional statistics, Machine Learning, International macroeconomics, and Finance.

\section{Education}
%
PhD Economics, \textbf{Aarhus University}, September 2012
\begin{innerlist}
	\item[Thesis title:] \emph{Large Panels and High Dimensional VARs.}
	\item[Topics:] Time-series, high-dimensional statistics, machine learning, macroneconomics.
	\item[Advisor:] Prof. Niels Haldrup.
	\item[Academic Visit:] \textbf{Princeton University}, Sponsors: Profs. Honor\'{e} and M\"{u}ller, Fall 2011.
\end{innerlist}
\vspace*{0.2cm}
M.Sc. Economics, \textbf{Aarhus University}, August 2009.
    \begin{innerlist}
		\item[Thesis title:] \emph{Modelling Exchange rates with Global VARs.}
		\item[Advisor:] Prof. Niels Haldrup.
    \end{innerlist}
\vspace*{0.2cm}
B.Sc. Economics, \textbf{University Paris X}, 2007.
\vspace*{0.2cm}
\newline
B.Sc. Mathematics, \textbf{University Paris VI}, 2007.
\begin{innerlist}
	\item[Minor:] Computer Science
\end{innerlist}





\section{Publications}

The latest version of the every paper with replication files is available on my website \href{https://lcallot.github.io/papers}{(link)}.

\begin{outerlist}
\item[] \textbf{Refereed Publications}\\
    \begin{innerlist}
\item[Callot, Caner, Kock, and Riquelme:] Sharp threshold detection based on sup-norm error rates in high-dimensional models" \textbf{Journal of Business \& Economic Statistics 2015}.
\item[Kock and Callot:] {Oracle Inequalities for High Dimensional Vector Autoregressions}. \textbf{Journal of Econometrics, 2015}.
\item[Callot and Kristensen:] {Regularized Estimation of Structural Instability in Factor Models: The US Macroeconomy and the Great Moderation}, \textbf{Advances in Econometris vol. 35, 2015.}
\item[Callot and Kock:] {Oracle Efficient Estimation and Forecasting with the Adaptive Lasso and the Adaptive Group Lasso in Vector Autoregressions}, \textbf{Essays in Nonlinear Time Series Econometrics (chapter 10) Oxford University Press, 2014.}
\item[Callot and Paldam:] {Natural funnel asymmetries. A simulation analysis of the three basic tools of meta analysis}, \textbf{Research Synthesis Methods, 2011}
\item[Callot, Haldrup, and Lamb:] {Deterministic and stochastic trends in the Lee-Carter mortality model}, \textbf{Applied Economics Letters, 2015}

    \end{innerlist}

\item[] \textbf{Working papers}\\
    \begin{outerlist}
\item[]Callot and Kristensen: {Vector Autoregressions with Parsimoniously Time-Varying Parameters and an Application to Monetary Policy}, \textit{Submitted}

\item[]Callot, Kock, Medeiros: {Estimation and Forecasting of Large Realized Covariance Matrices and Portfolio Choice}, \textit{Revision requested.}
\item[]Callot: {A Bootstrap Co-integration Rank Test for Panels of VAR Models}, \textit{CREATES working paper}
\item[]Callot: {Estimating and Testing for a Common Co-integration Space in Large Panel VAR}, \textit{CREATES working paper}
    \end{outerlist}


\section{Teaching Experience}
\begin{outerlist}
\item[] \textit{Lecturer}\\
    \begin{innerlist}
    	\item[]Introductory Econometrics, Spring 2013, 2014, and 2015. VU Amsterdam.
        \item[]Introduction to programming, August/September 2011. Aarhus University.
        \blankline
    \end{innerlist}

\item[] \textit{Teaching Assistant}\\
    \begin{innerlist}
  \item Business Mathematics, Spring 2013 and 2014.
  \item Introductory Statistics, Spring 2013 and 2014.
	\item[]Econometrics, Fall 2010 , Aarhus University.
	\item[]Advanced Programming in Quantitative Economics, August 2010 , Aarhus University.
        \item[]Macro 1, Fall 2009 , Aarhus University.
        \blankline
    \end{innerlist}


\item[] \textit{Research Assistant}\\
    \begin{innerlist}
        \item[]Assistant to Prof. Martin Paldam, 2008 and 2009.
        \blankline
    \end{innerlist}

\end{outerlist}


\section{Refereeing}
    \begin{innerlist}
    \item[]Journal of Business \& Economic Statistics
    \item[]Advances in Econometrics.
    \item[]Econometric Review.
    \item[]The Energy Journal.
    \item[]Computational Statistics.
    \end{innerlist}


\section{Invited seminars:}
    \begin{innerlist}
    \item[2015] The Tinbergen Institute, Amsterdam, the Netherlands.
    \item[2014] Seminar, Maastricht University, the Netherlands.
    \item[2012] Seminar, Uppsala University, Sweden.
    \end{innerlist}

\section{Research stays}
  \begin{innerlist}
    \item[2013] PUC Rio, Rio de Janeiro, Brazil.
    \item[2011] Princeton University, Princeton, USA.
  \end{innerlist}

\section{Languages}
%
\blankline
French (native), English (fluent), Danish (proficient), Spanish (basic).



% Hide in the academic version of the CV
\section{Private sector experience}

\begin{outerlist}
\item[2013-2014]Consultant, \href{http://danskecommodities.com/}{Danske Commodities}, Aarhus, Denmark. Short term forecasting of electricity demand imbalances.
\item[2005-2006] IT manager and statistician, SITELESC, Paris, France. Populating and management of a small business data base.
\item[2004-2005]IT technician, \href{www.lal.in2p3.fr}{Linear Accelerator Laboratory}, Paris, France. Hardware quality control and certification.
\end{outerlist}


% Hide in the academic version of the CV
\section{IT skills}
A list of languages, tools, statistical packages, and Operating Systems, classified by degree of proficiency.
\begin{innerlist}
\item[Expert:] R, parallel computing, Latex, Markdown
\item[Intermediate:] Linux (Debian, Gentoo, Ubuntu), Git, Matlab/Octave, SPSS, STATA
\item[Novice:] SQL, C, HTML, CSS, XML, Python.
\end{innerlist}

\end{document}

%%%%%%%%%%%%%%%%%%%%%%%%%% End CV Document %%%%%%%%%%%%%%%%%%%%%%%%%%%%%
